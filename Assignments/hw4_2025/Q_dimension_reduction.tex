{\bf [40 pts] dimensionality reduction}

In this problem, we will explore linear and nonlinear dimension reduction methods. Assume you have a high-dimensional dataset from gene expression, with each sample containing the expression levels of 1000 genes. The dataset is known to be divided into two main classes (e.g., tumor and normal cells), but their specific characteristics are not yet clear. Data can be find from provided\_data/gene\_expression.csv. Please answer the following questions. 



\begin{enumerate}[label=(\alph*)]
    \item Data Preparation and PCA. Standardize the gene expression data for each sample. After applying PCA, calculate the explained variance ratio of the first two principal components.
    \begin{solution}
     
    \end{solution}
    \item There are 2 key hyperparameters for t-SNE: perplexity and learning rate. Please briefly describe their potential impact on the dimensionality reduction results.
    \begin{solution}


    \end{solution}
    \item t-SNE Implementation. Implement t-SNE on the standardized gene expression data.
    Experiment with 3 different perplexity values (e.g., 5, 10, 50) and a fixed learning rate (200). Please visualize the results using scatter plots.
    Additionally, use a quantitative metric such as the silhouette score to evaluate which perplexity setting provides the best separation between “normal” and “tumor” samples.

    \begin{solution}

    \end{solution}   

\end{enumerate}





